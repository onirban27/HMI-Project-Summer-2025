\section{\textbf{Discussion}}

Overall, the analyses showed that feedback-based assistance prompted the most revisions and longest task durations, reflecting deeper iterative engagement, while example-based assistance achieved the highest effectiveness ratings and was the strongest predictor of overall success. Template-based support offered moderate benefits, improving outcomes relative to control but remaining less effective than the other two strategies. Control consistently lagged behind, with faster but less effective task performance.

\subsection {\textbf {Response to the Research Question}}
Structured assistance consistently outperformed no assistance across domains, with example-based and feedback-based strategies providing the strongest benefits. Example-based support was most effective for data analysis, while feedback-based support was most effective for educational content; in creative writing, both strategies outperformed templates and control, though differences were smaller. These findings indicate that assistance should be matched to task type, such as, examples are particularly beneficial for analytical tasks, feedback best supports instructional or explanatory tasks, and both reliably surpass templates and control.

\subsection {\textbf {Implications}}
The findings suggest that LLM-assisted interfaces should adopt domain aware defaults, offering example galleries for data analysis and iterative feedback tools for educational tasks. Since example and feedback based strategies are equally strong, systems should let users combine or switch between them, such as starting with examples and refining with feedback. Given the time quality trade-off of feedback, interfaces can frame it as a “deeper refinement mode” with time cues to manage effort. Support should also be confidence-adaptive: lighter templates for confident users, stronger helps for less confident ones. Finally, systems should evaluate both subjective effectiveness and behavioral signals to capture impact comprehensively and inform design.

\subsection {\textbf {Limitations}}
The within-subjects design enabled direct comparisons but introduced potential fatigue effects across the 12 conditions, which may have reduced participant engagement in later trials. The quality of LLM outputs was assessed through participants’ self-reported ratings, which reflect subjective perceptions rather than an objective scientific evaluation. Finally, the relatively small sample of 33 participants, drawn from a single educational and technical context, limits the generalizability of the findings to more diverse populations.

\subsection {\textbf {Future Work}}
Future research should address these limitations through several methodological enhancements. Implementing a mixed-methods approach combining quantitative effectiveness ratings with qualitative analysis of prompt revisions would illuminate the mechanisms through which different assistance conditions influence user behavior. Incorporating automated quality assessment of outputs would validate whether perceived effectiveness translates to measurable improvement. Larger and more diverse samples would enhance external validity as well. Future investigations should explore adaptive assistance systems that adjust support based on user confidence and task complexity. The optimal combinations of assistance conditions, such as beginning with example-based assistance and then transitioning to feedback-based refinement, might yield superior results to single-strategy approaches.

\subsection {\textbf {Conclusion}}
This study offers the systematic comparison of multiple assistance conditions for prompt revision across diverse LLM task domains, showing that structured support meaningfully improves outcomes over unguided revision. Beyond demonstrating the relative advantages of example and feedback-based approaches, the findings establish that even minimal scaffolds, such as templates, provide value for certain users and contexts. These results extend the growing literature on human–LLM interaction by highlighting the importance of aligning assistance design with task characteristics. Overall, the study underscores that effective prompt revision is not simply a matter of user persistence, but of providing the right kind of structured support to unlock the full potential of LLMs.
